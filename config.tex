% ---
% Pacotes básicos
% ---
\usepackage{lmodern}			% Usa a fonte Latin Modern
\usepackage[T1]{fontenc}		% Selecao de codigos de fonte.
\usepackage[utf8]{inputenc}		% Codificacao do documento (conversão automática dos acentos)
\usepackage{lastpage}			% Usado pela Ficha catalográfica
\usepackage{indentfirst}		% Indenta o primeiro parágrafo de cada seção.
\usepackage{color}				% Controle das cores
\usepackage{graphicx}			% Inclusão de gráficos
\usepackage{microtype} 			% para melhorias de justificação
\usepackage{tikz}
\usepackage{wrapfig}
% add to latexmk > http://tex.stackexchange.com/questions/105943/latexmk-and-nomencl
\usepackage[portuguese]{nomencl} % \nomenclature{}{}
\makenomenclature
% ---
\newcommand{\subfigureautorefname}{\figureautorefname} % resolve problema autoref subfig

\usepackage[labelfont=bf]{caption}

\makeatletter
\@ifundefined{showcaptionsetup}{}{%
 \PassOptionsToPackage{caption=false}{subfig}}
\usepackage{subfig}
\makeatother

% ---
% Pacotes adicionais, usados apenas no âmbito do Modelo Canônico do abnteX2
% ---
\usepackage{lipsum}				% para geração de dummy text
% ---

% ---
% Pacotes de citações
% ---
\usepackage[brazilian,hyperpageref]{backref}	 % Paginas com as citações na bibl
\usepackage[alf,abnt-emphasize=bf]{abntex2cite}	% Citações padrão ABNT

% ---
% CONFIGURAÇÕES DE PACOTES
% ---

% ---
% Configurações do pacote backref
% Usado sem a opção hyperpageref de backref
\renewcommand{\backrefpagesname}{Citado na(s) página(s):~}
% Texto padrão antes do número das páginas
\renewcommand{\backref}{}
% Define os textos da citação
\renewcommand*{\backrefalt}[4]{
	\ifcase #1 %
		Nenhuma citação no texto.%
	\or
		Citado na página #2.%
	\else
		Citado #1 vezes nas páginas #2.%
	\fi}%
% ---

% ---
% Informações de dados para CAPA e FOLHA DE ROSTO
% ---
\titulo{Desenvolvimento de um sistema de resposta para uso em sala de aula}
\autor{Pedro Henrique Araújo Sobral}
\local{Juazeiro, Bahia, Brasil}
\data{2017}
\orientador{Dr. Max Santana Rolemberg Farias}
% \coorientador{João Carlos Sedraz Silva}
% \instituicao{%
%   \textsc{Universidade Federal do Vale do São Francisco} -- UNIVASF
%   \par
%   \textsc{Colegiado de Engenharia de Computação}
%  % \par
%  }
\tipotrabalho{Trabalho de Conclusão de Curso}
% O preambulo deve conter o tipo do trabalho, o objetivo,
% o nome da instituição e a área de concentração
\preambulo{
Trabalho de conclusão de curso \mbox{apresentado} à
Universidade Federal do Vale do São \mbox{Francisco},
Campus Juazeiro -- BA, como requisito para a obtenção do
título de Engenheiro de \mbox{Computação}.}
% ---


% ---
% Configurações de aparência do PDF final

% alterando o aspecto da cor azul
\definecolor{blue}{RGB}{41,5,195}

% informações do PDF
\makeatletter
\hypersetup{
   	%pagebackref=true,
		pdftitle={\@title},
		pdfauthor={\@author},
  	pdfsubject={\imprimirpreambulo},
    pdfcreator={LaTeX with abnTeX2},
		pdfkeywords={abnt}{latex}{abntex}{abntex2}{trabalho acadêmico},
		colorlinks=true,       		% false: boxed links; true: colored links
  	linkcolor=blue,%blue,          	% color of internal links
  	citecolor=blue,%blue,        		% color of links to bibliography
  	filecolor=magenta,%magenta,      		% color of file links
		urlcolor=blue,%blue,
		bookmarksdepth=4
}
\makeatother
\settocdepth{subsection}
% ---

% ---
% Espaçamentos entre linhas e parágrafos
% ---

% O tamanho do parágrafo é dado por:
\setlength{\parindent}{1.3cm}

% Controle do espaçamento entre um parágrafo e outro:
\setlength{\parskip}{0.2cm}  % tente também \onelineskip

% ---
% compila o indice
% ---
% \makeindex
% --
\usepackage{lmodern}

\newcommand{\clicker}{\textit{clicker}}
\newcommand{\clickers}{\textit{clickers}}
