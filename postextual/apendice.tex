% ---
% Inicia os apêndices
% ---
\begin{apendicesenv}

% Imprime uma página indicando o início dos apêndices
\partapendices

% ----------------------------------------------------------

% ----------------------------------------------------------

\chapter{Questionário SUS - System Usability Scale}\label{appendix:sus}

Questionário SUS aplicado aos participantes dos testes de usabilidade.
Versão em língua portuguesa desenvolvida por \cite{tenorio2010desenvolvimento} que passou por um processo
de tradução.

Cada um dos itens do questionário era composto por alternativas na escala de Likert como a seguir:
\begin{enumerate}
  \item Discordo totalmente.
  \item Discordo parcialmente.
  \item Não concordo, nem discordo.
  \item Concordo parcialmente.
  \item Concordo totalmente.
\end{enumerate}
Questionário:
\begin{enumerate}
  \item
  Eu acho que gostaria de usar esse sistema frequentemente.
  \item
  Eu acho o sistema desnecessariamente complexo.
  \item
  Eu achei o sistema fácil de usar.
  \item
  Eu acho que precisaria de ajuda de uma pessoa com conhecimentos técnicos para usar o sistema.
  \item
  Eu acho que as várias funções do sistema estão muito bem integradas.
  \item
  Eu acho que o sistema apresenta muita inconsistência.
  \item
  Eu imagino que as pessoas aprenderão como usar esse sistema rapidamente.
  \item
  Eu achei o sistema atrapalhado de usar.
  \item
  Eu me senti confiante ao usar o sistema.
  \item
  Eu precisei aprender várias coisas novas antes de conseguir usar o sistema.
\end{enumerate}

\chapter{Tarefas - Testes de Usabilidade}\label{appendix:tasks}

\begin{enumerate}
  \item \textbf{Tarefa \#1 Criar uma nova turma}
  \begin{itemize}
    \item Suponha que você seja o professor da disciplina Matemática Discreta. Você deseja criar uma nova turma.
    \begin{itemize}
      \item Tente criar a turma Mat. Discreta com código de acesso DISC17.
      \item Ative a turma que você acabou de criar.
    \end{itemize}
  \end{itemize}

  \item \textbf{Tarefa \#2 Adicionar estudantes em uma turma}
  \begin{itemize}
    \item Agora que você criou a turma DISC17, você deseja inserir estudantes para que tenham acesso a nova turma criada.
    \begin{itemize}
      \item Tente adicionar o estudante de nome Pedro com ID 101 e o estudante Luiz com ID 102.
    \end{itemize}
  \end{itemize}

  \item \textbf{Tarefa \#3 Criar uma nova questão}
  \begin{itemize}
    \item Com a turma criada e estudantes adicionados, você como professor quer adicionar questões para perguntar aos alunos na sala de aula.
    \begin{itemize}
      \item Você deve tentar adicionar uma nova questão de verdadeiro e falso e indicar que ela é FALSA. Salve.
      \item Tente criar outra questão de múltipla escolha com 4 alternativas. Salve.
    \end{itemize}
  \end{itemize}

  \item \textbf{Tarefa \#4 Realizar frequência dos estudantes}

  \begin{itemize}
    \item Agora você está preparado para a sua aula. Na sala de aula você quer realizar a frequência dos estudantes pelo sistema.
    \begin{itemize}
      \item Inicie o processo para realizar frequência dos estudantes pelo sistema.
      \item Siga a próxima tarefa.
    \end{itemize}
  \end{itemize}

  \item \textbf{Tarefa \#5 Acessar a sala como estudante}
  \begin{itemize}
    \item Suponha que você é aluno da turma de Discreta com código de acesso DISC17 e o seu código de identificação é 101.
    \begin{itemize}
      \item Você chegou na sala de aula.
      \item Tente fazer o login no sistema.
    \end{itemize}
  \end{itemize}

  \item \textbf{Tarefa \#6 Responder a frequência}
  \begin{itemize}
    \item O professor já iniciou o processo para realizar a frequência da sala pelo aplicativo.
    \begin{itemize}
      \item Tente realizar a frequência pelo aplicativo.
      \item Encerre a frequência no aplicativo do professor.
    \end{itemize}
  \end{itemize}

  \item \textbf{Tarefa \#7 Iniciar sessão de perguntas}
  \begin{itemize}
    \item Chegou o momento de apresentar algumas questões para os estudantes.
    \begin{itemize}
      \item Tente iniciar uma sessão com as perguntas que você criou.
      \item Ative uma questão para votação e veja o que acontece no aplicativo do estudante.
      \item Siga para a próxima tarefa.
    \end{itemize}
  \end{itemize}

  \item \textbf{Tarefa \#8 Responder questão}
  \begin{itemize}
    \item A questão foi disponibilizada pelo professor para ser respondida.
    \begin{itemize}
      \item Responda a questão pelo aplicativo
      \item Veja o que acontece na tela do professor.
    \end{itemize}
  \end{itemize}

  \item \textbf{Tarefa \#9 Finalizar sessão de perguntas}
  \begin{itemize}
    \item Chegou o momento de apresentar algumas questões para os estudantes.
    \begin{itemize}
      \item Pause a votação
      \item Mostre qual a resposta correta
      \item Mostre o resultado da votação.
    \end{itemize}
  \end{itemize}

  \item \textbf{Tarefa \#10 Encerrar sessão de perguntas}
  \begin{itemize}
    \item Suponha que a aula acabou.
    \begin{itemize}
      \item Tente finalizar a sessão de questões.
    \end{itemize}
  \end{itemize}

  \item \textbf{Tarefa \#11 Verificar resposta dos estudantes}
  \begin{itemize}
    \item Suponha que agora você quer saber como cada estudante respondeu a questões.
    \begin{itemize}
      \item Veja se você consegue obter essa informação no sistema.
    \end{itemize}
  \end{itemize}
\end{enumerate}

\end{apendicesenv}
% ---
