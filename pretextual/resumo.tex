% resumo em português
\setlength{\absparsep}{18pt} % ajusta o espaçamento dos parágrafos do resumo
\begin{resumo}
  Sistemas de resposta em sala de aula permitem ao professor um retorno
  em tempo real sobre o entendimento de toda uma classe sobre um determinado
  tópico de estudo. Essa informação é valiosa porque permite ao educador, por exemplo,
  realizar uma avaliação formativa, orientando-o na prática pedagógica.
  No entanto, o custo associado à aquisição dos sistemas de resposta podem ser proibitivos ao uso.
  % É importante destacar essa tecnologia apenas como um meio para colaborar no processo
  % de ensino e aprendizagem, e só faz sentido quando associada com práticas
  % pedagógicas de ensino como o aprendizado ativo.
  Desta forma, neste trabalho foi desenvolvido um sistema de resposta em sala de aula,
  que possibilite aos professores e aos estudantes uma ferramenta de software livre, que permita
  usar \textit{smartphones} como \textit{clickers}.
  O sistema desenvolvido foi avaliado por professores em testes de usabilidade, identificando
  pontos de melhoria em seu design. Com algumas melhorias realizadas, o sistema foi então
  aplicado e avaliado em um ambiente de ensino com 12 estudantes.
  Tanto pelos professores, como pelos estudantes, a solução apresentou bons
  índices de usabilidade, com uma percepção otimista com relação ao uso do
  sistema na sala de aula como ferramenta capaz de contribuir no processo de ensino e aprendizagem
  dos estudantes.

 \textbf{Palavras-chave}:  Aprendizado Ativo. Instrução pelos Colegas. Sistemas de Resposta em Sala de Aula. Aplicativos Híbridos para Celular.
\end{resumo}

% resumo em inglês
\begin{resumo}[Abstract]
 \begin{otherlanguage*}{english}
  %  Classroom response systems (CRS) allow the teacher a real-time feedback on the understanding of a whole class on a topic of study. This information is valuable because it allows the educator, for example, conduct a formative assessment, guiding it in pedagogical practice.
   %
  %  However, it is important to highlight this technology only as a means to assist in the process of teaching and learning, and only makes sense when combined with pedagogical practices of teaching and active learning.
  %  Moreover, the cost associated with clickers can be prohibitive to use. Thus, this work will develop a CRS that allows teachers and students a free software that will give them the opportunity to use smartphones as clickers so that it can be mainly used for educational practices of active learning as Peer Instruction.
   %
% Classroom response systems allow the teacher a real-time feedback on an entire class's understanding of a particular topic of study. This information is valuable because it allows the educator, for example, to carry out a formative evaluation, guiding it in pedagogical practice.
   Classroom response systems (CRS) allow the teacher a real-time feedback on the understanding of a whole class on a topic of study. This information is valuable because it allows the educator, for example, conduct a formative assessment, guiding it in pedagogical practice.
However, the cost associated with acquiring response systems may be prohibitive to use. In this work, a classroom response system was developed that allows teachers and students to use a free software tool to use smartphones as clickers. The developed system was evaluated by teachers in usability tests, identifying improvement points in their design. With some improvements made, the system was then applied and evaluated in a teaching environment with 12 students. Both by teachers and students, the solution presented good usability scores, with an optimistic perception regarding the use of the system in the classroom as a tool capable of contributing to the teaching and learning process.



   \noindent
   \textbf{Keywords}: Active Learning. Peer Instruction. Classroom Response Systems. Hybrid Mobile Applications.
 \end{otherlanguage*}
\end{resumo}
