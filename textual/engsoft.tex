\clearpage
\section{Engenharia de Software}

Engenharia de Software pode ser definida como:
\begin{citacao}[english]
1. the systematic application of scientific and technological knowledge, methods, and experience to the design,
implementation, testing, and documentation of software [...]  2. the application
of a systematic, disciplined, quantifiable approach to the development, operation,
and maintenance of software; that is, the application of engineering to software \cite{IEEE2010}.
\end{citacao}

A engenharia de software deve ter foco na qualidade, que apoia as outras camadas
dessa tecnologia, que são as camadas de processo, métodos e ferramentas \autoref{fig:desen_engsoft}.
A camada de processo define um conjunto de atividades ou um arcabouço que tem
como finalidade garantir a efetiva utilização da tecnologia engenharia de software, que então
leva à produção de um produto de software. Os detalhes de como fazer o software pertencem
a camada de métodos. Os métodos da enghenharia de software incluem tarefas de planejamento
e estimativa de software, análise de requisitos, modelagem de projeto, codificação,
testes e manutenção. As ferramentas de engenharia de software auxiliam as camadas de
processo e métodos, com ferramentas automatizadas, que por sua vez, quando integradas,
é estabelecido um suporte ao desenvolvimento de software chamado CASE -
\textit{Computer Aided Software Engineering} \cite{Pressman2009, Sommerville2006}.


\begin{figure}[htbp]
  \centering
  \caption{Engenharia de Software - uma tecnologia em camadas}
  \includegraphics[scale=0.45]{imagens/desenv_engsoft2}
  \label{fig:desen_engsoft}
  \fonte{\cite{Pressman2009}}
\end{figure}
