% resumo em português
\setlength{\absparsep}{18pt} % ajusta o espaçamento dos parágrafos do resumo
\begin{resumo}
  Sistemas de resposta em sala de aula ou {\clickers} permitem ao professor um retorno
  em tempo real sobre o entendimento de toda uma classe sobre um determinado
  tópico de estudo. Essa informação é valiosa porque permite ao educador, por exemplo,
  realizar uma avaliação formativa, orientando-o na prática pedagógica. No entanto,
  é importante destacar essa tecnologia apenas como um meio para colaborar no processo
  de ensino e aprendizagem, e só faz sentido quando associada com práticas
  pedagógicas de ensino como o aprendizado ativo. Por outro lado, o custo associado
  aos {\clickers} podem ser proibitivos ao uso. Dessa forma, nesse trabalho será
  desenvolvido um sistema de resposta em sala de aula, que possibilite aos
  professores e aos estudantes uma ferramenta de software livre, que permita
  usar \textit{smartphones} como {\clickers} para que o mesmo possa ser usado
  principalmente com práticas pedagógicas de aprendizado significativo como a Instrução pelo Colegas.

 \textbf{Palavras-chave}:  Aprendizado Ativo. Instrução pelos Colegas. Sistemas de Resposta em Sala de Aula. Aplicativos Híbridos para Celular.
\end{resumo}

% resumo em inglês
\begin{resumo}[Abstract]
 \begin{otherlanguage*}{english}
   Classroom response systems (CRS) or just clickers allow the teacher a real-time feedback on the understanding of a whole class on a topic of study. This information is valuable because it allows the educator, for example, conduct a formative assessment, guiding it in pedagogical practice.
   However, it is important to highlight this technology only as a means to assist in the process of teaching and learning, and only makes sense when combined with pedagogical practices of teaching and active learning. Moreover, the cost associated with clickers can be prohibitive to use. Thus, this work will develop a CRS that allows teachers and students a free software that will give them the opportunity to use smartphones as clickers so that it can be mainly used for educational practices of active learning as Peer Instruction.

   \noindent
   \textbf{Keywords}: Active Learning. Peer Instruction. Classroom Response Systems. Hybrid Mobile Applications.
 \end{otherlanguage*}
\end{resumo}
