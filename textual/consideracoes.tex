\chapter{Considerações Finais e Trabahos Futuros}\label{chap:final_thoughts}

\section{Considerações Finais}

Este trabalho teve como objetivo principal o desenvolvimento de um sistema de resposta para uso em sala de aula,
que possibilita-se aos professores e aos estudantes uma ferramenta de software livre,
que permitesse usar {\textit{smartphones}} como \textit{clickers}.

Nesse sentido, os resultados obtidos são animadores ao apontar que a solução testada alcançou bons índices
de usablididade tanto pelos professores, quanto pelo estudantes. Além disso, a percepção otimista
do professor voluntário no experimento do uso do sistema na sala de aula e o interesse dele
em continuar a usar o sistema em suas aulas reforçam que o objetivo foi alcançado, contribuindo assim
com a comunidade acadêmica com um software livre - ainda que no início do seu desenvolvimento - de qualidade.

\section{Trabalhos Futuros}

Acredita-se que a melhor forma de continuar esse trabalho é utilizando o sistema
desenvolvido em um turma por um longo período para assim descobrir e desenvolver outros requistios que não
foram considerados nesse trabalho. Além disso, pode-se fazer um estudo de caso para verificar os benefícios
do sistema na sala de aula, ou seja, aqueles que foram descritos no \autoref{chap:revision}, \autoref{sec:beneficios},
como aumento na frequência escolar, melhora na atenção, engajamento da turma, etc.
