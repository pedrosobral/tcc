\chapter{Introdução}

Por que ainda se justifica o modelo de ensino em que se baseia
na disseminação de informações, que nunca foi tão fácil achar (internet,
livros, etc)? O ensino de hoje deveria estar focado para uma nova
visão em que o papel do professor seja de intermediar a aprendizagem \cite{Araujo2013}.

Nesse modelo de ensino, o construtivismo, os estudantes são agentes ativos do próprio processo
de ensino e construção do conhecimento. Uma aplicação prática desse modelo teórico de ensino é o aprendizado ativo.
O aprendizado ativo é um conjunto de práticas pedagógicas que além de envolver os estudantes no
fazer, os faça pensar no que estão fazendo, ou seja, além de ouvir, eles precisam ler,
escrever, discutir, ou estarem envolvidos na resolução de problemas \cite{Charles1991}.

Uma metodologia de aprendizado ativo que tem alcançado sucesso internacionalmente é
a Instrução pelos Colegas (IpC) ou \textit{peer instruction} \cite{Araujo2013}. Resumidamente,
o IpC promove o aprendizado ativo com questionamentos regulares na aula, fazendo com que os
estudantes passem mais tempo discutindo e pensando sobre um conteúdo, do que assistindo passivamente a aula.

Assim como no modelo de ensino tradicional, a tecnologia também pode ser usada para promover
de forma mais eficiente o aprendizado ativo. No caso do IpC, a etapa fundamental de votação de uma questão
(apresentado em detalhes no \autoref{chap:revision}, \autoref{section:ipc}),
o uso de sistemas de resposta permitem ao professor uma análise posterior dos resultados,
além de dinamizar o processo de votação em sala de aula.

Sistemas de respostas são sistemas que possibilitam que todos os alunos
respondam a questões apresentadas pelo professor. Geralmente um gráfico de barras
é apresentado logo em seguida que os estudantes submetem as suas soluções
utilizando algum dispositivo remoto também conhecido como \textit{clickers}. As respostas são anônimas para os seus colegas,
no entanto o professor pode identificar cada estudante individualmente pela
identificação única do dispositivo, permitindo assim uma análise individual \cite{Kay2009}.
% O resultado imediato disponibilizado por tais sistemas pode ser usado juntamente
% com metodos que promovem a interação social voltada para a aprendizagem como o IpC.

Não existe na literatura um consenso sobre a nomenclatura para referenciar sistemas de
resposta para uso em sala de aula. Pode-se encontrar termos como
sistema de resposta do estudante \textit{``student response system''},
sistema de resposta a audiência \textit{``audience response system''},
sistema de resposta pessoal \textit{``personal response systems''},
sistemas de resposta em sala de aula {\textit{``classroom response systems''} (CRS)\nomenclature{CRS}{Classroom Response System}},
ou apenas como \textit{``clickers''} \cite{Hunsu2016}.
Nesse trabalho, para simplificação foi utilizado apenas a sigla {\clicker} {\textit{``Classroom Response System''}
para referenciar esse tipo de sistema.

O uso de CRS associado com um modelo de aprendizado ativo tem demonstrado
trazer vários benefícios para a sala de aula, como aumento na frequência escolar \cite{Fotaris2016},
melhora na atenção dos estudantes \cite{Terrion2012} e maior engajamento da turma \cite{Kaya2016}.
Também promove benefícios para a aprendizagem com um aumento na interação e discussão na sala de aula \cite{Mattos2015, Barragues2011},
e com melhora no aprendizado \cite{sun2014, Hunsu2016}. Além dos
benefícios para avaliação pelo \textit{feedback} imediato do entendimento dos estudantes
sobre um determinado conteúdo  \cite{Rana2016, Blood2013}.

Apesar dos benefícios significativos do uso associado com CRS na sala de aula,
o preço de tais tecnologias podem representar um custo econômico considerável
para instituições de ensino, que pode se tornar uma barreira para adoção de CRS e
adotá-las no processo de aprendizagem \cite{Blasco-Arcas2013}.

Nesse sentido, considerando que os \textit{smartphones} podem ser usados
como parte integrante dos CRSs como o dispositivo em que os estudantes vão submeter as respostas (ou \textit{clickers}),
e da quase ubiquidade de tais dispositivos entre os jovens brasilieiros - já em 2014 mais de 93\% dos estudantes da rede
privada de ensino e quase 65\% dos da rede pública possuem \textit{smartphones} \cite[p. 55]{IBGE2016} -,
este trabalho apresenta o desenvolvimento de um CRS de software livre que permita usar
{\textit{smartphones}} como \textit{clickers}.

\section{Objetivo Geral}
Desenvolver um sistema de resposta para uso sala de aula (CRS),
que possibilite aos professores e aos estudantes uma ferramenta de software livre,
que permita usar {\textit{smartphones}} como \textit{clickers}.

\section{Objetivos Específicos}

\begin{itemize}
    \item Realizar levantamento de requisitos sobre os sistemas de resposta em sala de aula;
    \item Especificar e implementar uma aplicação para dispositivos móveis, que será utilizado como \textit{clickers};
    \item Especificar e implementar uma aplicação web para o professor administrar as questões e gerar relatórios;
    \item Especificar e implementar um sistema servidor, para receber e
    enviar dados para os os clientes: dispositivos móveis dos alunos e navegador
    web do professor.
\end{itemize}

\section{Organização do texto}
Além desta introdução, este trabalho está dividido em mais quatro capítulos.

\begin{description}
  \item[Revisão de literatura:] Esse capítulo discute um contexto pedagógico
  para o uso de sistemas de resposta em sala de aula. Inicialmente, é
  apresentado o conceito de aprendizado ativo e um método de implementação: Instrução pelos Colegas.
  Em seguida, são apontados os sistemas de resposta em sala de aula como uma
  ferramenta para ajudar o professor a mediar um aprendizado significativo  em
  sala de aula. O capítulo é encerrado apresentando os benefícios e desafios de uso dessa tecnologia.

  \item[Engenharia de Software:] Nesse capítulo são descritas as fases de especificação, projeto, implementação e testes do sistema.
  Os procedimentos, material e métodos, resultados e discussões e cada etapa também é apresentado nesse capítulo.

  \item[Resposta em Sala de Aula:] Nesse capítulo é apresentado o software desenvolvido,
  definições, as principais telas e característica do sistema desenvolvido.

  \item[Considerações finais e trabalhos futuros:] São realizadas as considerações finais do
  trabalho e apontadas diretrizes para trabalhos futuros.

\end{description}
