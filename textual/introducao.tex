\chapter{Introdução}

Assim como o artista propõe a sua obra ao público, assim deve ser o professor que
propõe conhecimento aos seus estudantes. No entanto, hoje parece que ainda prevalece
o modelo em que o professor é o transmissor de informações em aulas puramente expositivas,
em que prevalece a baixa participação dos alunos \cite[p. 8]{Silva2001}.

Por que ainda se justifica o modelo de ensino em que se baseia
na disseminação de informações, que nunca foi tão fácil achar (internet,
livros, etc)? O ensino de hoje deveria estar focado para uma nova
visão em que o papel do professor seja de intermediar o ensino \cite[p. 19]{Araujo2013}.

Sistemas de respostas são sistemas que possibilitam que todos os alunos
respondam a questões apresentadas no projetor. Geralmente um gráfico de barras
é apresentado logo em seguida que os estudantes submetem as suas soluções
utilizando algum dispositivo remoto. As respostas são anônimas para os seus colegas,
no entanto o professor pode identificar cada estudante individualmente pela
identificação única do dispositivo, permitindo assim uma análise individual \cite[p. 1]{Kay2009}.


Esse resultado imediato pode ser usado juntamente
com metodos que promovem a interação social voltada para a aprendizagem como a
Instrução pelos Colegas (IpC) que tem alcançado sucesso internacionalmente \cite[p. 3]{Araujo2013}.

\section{Objetivo Geral}
Desenvolver um sistema de resposta interativo
para auxiliar o ensino-aprendizagem no meio acadêmico.

\section{Objetivos Específicos}

\begin{itemize}
    \item Levantamento do referencial te\'orico sobre os sistemas de resposta
    interativos;
    \item Realizar levamtamento de requisitos do sistema;
    \item Projetar e implementar um banco de dados \ldots;
\end{itemize}

\section{Organização do texto}
