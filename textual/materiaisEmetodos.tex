\chapter{Material e Métodos}
\begin{quote}\normalfont\itshape\vspace*{-2\baselineskip}
Neste capítulo serão descritas as fases de especificação e projeto do sistema. Ainda nesse capítulo são apresentadas as ferramentas e a arquitetura do sistema, dando uma visão geral do que foi desenvolvido.
\end{quote}

\section{Engenharia de Software}

Engenharia de Software pode ser definida como:
\begin{citacao}[english]
1. the systematic application of scientific and technological knowledge, methods, and experience to the design,
implementation, testing, and documentation of software [...]  2. the application
of a systematic, disciplined, quantifiable approach to the development, operation,
and maintenance of software; that is, the application of engineering to software \cite{IEEE2010}.
\end{citacao}

\begin{figure}[!b]
  \centering
  \caption{Engenharia de Software - uma tecnologia em camadas}
  \includegraphics[scale=0.33]{imagens/desenv_engsoft2}
  \label{fig:desen_engsoft}
  \fonte{\cite{Pressman2009}}
\end{figure}

A engenharia de software deve ter foco na qualidade, que apoia as outras camadas
dessa tecnologia, que são as camadas de processo, métodos e ferramentas \autoref{fig:desen_engsoft}.
A camada de processo define um conjunto de atividades ou um arcabouço que tem
como finalidade garantir a efetiva utilização da tecnologia engenharia de software, que dessa forma
leva à produção de um software. Os detalhes de como fazer o software pertencem
a camada de métodos. Os métodos da engenharia de software incluem tarefas de planejamento
e estimativa de software, análise de requisitos, modelagem de projeto, codificação,
testes e manutenção. As ferramentas de engenharia de software auxiliam as camadas de
processo e métodos, com ferramentas automatizadas, que por sua vez, quando integradas,
é estabelecido um suporte ao desenvolvimento de software chamado CASE\nomenclature{CASE}{Computer Aided Software Engineering} -
\textit{Computer Aided Software Engineering} \cite{Pressman2009, Sommerville2006}.

Entre o conjunto de atividades definidas pela camada de processo, quatro são
fundamentais, a saber, especificação de software, projeto e implementação de
software, validação de software e evolução de software. Especificação de software
ou engenharia de requisitos é uma fase importante e crítica do processo de engenharia
de software. Importante porque é uma análise de requisitos bem feita que possibilitará
atendar as demandas dos usuários. Crítica porque um sistema mal especificado, pode até ser
bem projetado e construído, mas não vai atender as necessidades dos usuários.
Em seguida, na fase de projeto e implementação os requisitos são projetados e programados,
tendo como resultado um sistema executável. Depois, o software deve ser verificado
para mostrar que atende às demandas dos usuários (validação do software). Finalmente,
na fase de evolução de software, o mesmo é modificado devido às mudanças
de requisitos e às necessidades dos usuários.

\subsection{Especificação do Sistema}
Nessa primeira etapa do projeto, foi utilizada a ferramenta \textit{Astah Community} \cite{astah2016}
para a criação de documentação em linguagem de modelagem unificada (UML)\nomenclature{UML}{Unified Modeling Language}. Os diagramas
de casos de uso UML são largamente utilizados para especificação de requisitos \cite{Sommerville2006}.
A \autoref{fig:usecases} exemplifica os casos de uso do sistema, fornecendo uma
visão geral do mesmo.

\begin{figure}[!htb]
  \centering
  \caption{Diagrama de Caso de Uso: Visão geral do sistema}
  \includegraphics[width=.75\textwidth]{imagens/casodeuso}
  \fonte{Elaborado pelo autor}
  \label{fig:usecases}
\end{figure}

\subsection{Projeto e Implementação}
O diagrama de implantação UML ou \textit{deployment diagram} mostra o \textit{hardware}
do sistema ou elementos de processamento, os componentes de software instalados
no \textit{hardware} e o \textit{middleware} usado para conectar os diferentes
nós do sistema \cite{Pressman2009}. A \autoref{fig:deployment_diagram} mostra
o diagrama de implantação do sistema que será desenvolvido. É possível observar
uma solução simples, mas que faz uso de soluções \textit{open-source} e de
qualidade reconhecida como será descrito a seguir.

\subsubsection{Plataforma}

\begin{figure}[!b]
  \centering
  \caption{Diagrama de Implantação do sistema}
  \includegraphics[width=1\textwidth]{imagens/deployment_diagram}
  \fonte{Elaborado pelo autor}
  \label{fig:deployment_diagram}
\end{figure}

\begin{description}
  \item[Servidor HTTP Apache]
  o Servidor HTTP\nomenclature{HTTP}{Hypertext Transfer Protocol} Apache (``httpd'') foi lançado em 1995 e desde abril de 1996 é o
  servidor web \textit{open-source} mais popular do mundo \cite{apache2016}.
  Segundo \citeonline{W3Techs2016} o Apache é usado por 51,3\% dos sites ativos no mundo.

  \item[PostgreSQL] sistema gerenciador de banco de dados objeto relacional (SGBDOR).
  Entre os SGBDORs\nomenclature{SGBDORs}{Sistema Gerenciador de Banco de Dados Objeto Relacional}
  \textit{open-source} mais avançados, totalmente compatível com
  o padrão ANSI-SQL:2008\nomenclature{ANSI-SQL}{American National Standards Institute - Structured Query Language},
  com recursos de integridade de dados includindo chaves
  primárias compostas, chave estrangeira, restrições \textit{check, unique, not null},
  gatilhos (\textit{triggers}), visões (\textit{views}), \textit{storage procedures}
  e recursos avançados como colunas de auto-incremento por exemplo.
  O PostgreSQL é distribuído sobre a licença permissiva MIT \cite{postgree2016}.

  \item[Python] uma das linguagens de programação mais populares do mundo,
  figurando entre as 5 linguagens mais utilizadas \cite{TIOBE2016, RedMonk2016, PYPL2016}.
  A escolha pela tecnologia Python como linguagem de programação se deu principalmente
  porque um dos principais \textit{frameworks} para o desenvolvimento web \textit{Django} \cite{Django2016},
  é escrito em Python. O \textit{Django} conta com mapeador objeto-relacional,
  interface de administração automática, URL's\nomenclature{URL}{Uniform Resource Identifier} elegantes, sistema de templates e
  sistema de cache \cite{Django2016}.

  \item[Ionic] é um \textit{framework open-source} para o desenvolvimento de aplicativos
  híbridos utilizando tecnologias web como HTML\nomenclature{HTML}{HyperText Markup Language},
  CSS\nomenclature{CSS}{Cascading Style Sheets} e JavaScript otimizadas
  para dispositivos móveis, com código fonte sobre a licença MIT\nomenclature{MIT}{Massachusetts Institute of Technology} \cite{ionic2016}.
  Uma das principais vantagens do desenvolvimento de aplicativos híbridos é que com
  apenas um código base é possível criar aplicativos para várias plataformas como
  iOS, Android e Windows Phone, que aliás foi uma das razões que fez o Moodle
  usar o Ionic como \textit{framework} para o desenvolvimento do \textit{Moodle Mobile 2} \cite{moodle2016}.

  \item[WebSocket] é um protocolo que possibilita abrir um canal interativo de comunicação
  entre o navegador e o servidor. Na verdade, esse canal é bidirecional (\textit{full-duplex})
  que utiliza apenas um soquete TCP\nomenclature{TCP}{Transmission Control Protocol} \cite{websocket2016}. A tecnologia \textit{WebSocket} será usada
  para permitir votação e controle de frequência em tempo-real.
\end{description}

\section{Cronograma de atividades}

\begin{table}[!hb]
  \centering
  % \caption{Cronograma de atividades para o TCC II}
  \begin{tabular}{|l|c|c|c|c|c|}
  \hline
  Atividade & Set/16 & Out/16 & Nov/16 & Dez/16 & Jan/17\tabularnewline
  \hline
  Correções da Banca & x &  &  &  & \tabularnewline
  \hline
  Revisão Literatura & x & x & x & x & \tabularnewline
  \hline
  Implementação & x & x & x & x & \tabularnewline
  \hline
  Conclusões &  &  &  & x & x\tabularnewline
  \hline
  Entrega/Defesa TCC II &  &  &  &  & x\tabularnewline
  \hline
  Correções TCC II &  &  &  &  & x\tabularnewline
  \hline
  \end{tabular}
\end{table}
