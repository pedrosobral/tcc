\chapter{Considerações Finais e Trabahos Futuros}

\section{Considerações Finais}

Falar um pouco sobre os CRS e importancia.

Este trabalho teve como objetivo principal o desenvolvimento de um CRS,
que possibilita-se aos professores e aos estudantes uma ferramenta de software livre,
que permitesse usar {\textit{smartphones}} como \textit{clickers}.

excelente escore de usabilidade tanto pelo professores como pelos estudantes

percepeção do professor do teste em sala de aula

% Esta pesquisa teve como questão norteadora a análise do uso das tecnologias móveis no contexto de aprendizagem de aulas de campo de Geologia. Nesse sentido, os resultados obtidos são animadores ao apontar que a solução testada alcançou bons índices de usabilidade e utilidade percebidas pelos estudantes. Somando a isso, a percepção otimista do professor participante do estudo de caso e o interesse dele em continuar a usar o sistema em suas práticas de campo reforçam a conclusão sobre a boa aceitação das tecnologias móveis nesse contexto de aprendizagem e indicam que elas apresentam potencial para se tornarem importantes ferramentas de apoio em aulas de campo.
% Com relação aos problemas e aspectos negativos apontados pelos alunos, as queixas principais foram em torno da necessidade de ampliação de recursos da aplicação testada. Este fato não contradiz a importância das tecnologias móveis em campo, apenas demonstra que mais estudos precisam ser feitos para se identificar as ferramentas móveis suficientes para os diferentes contextos de aprendizagem nas quais elas podem ser utilizadas.
% Como trabalho futuro, pretende-se realizar experimentos de m-learning em aulas de campo de outras áreas, como Biologia e Geografia, para se identificar as diferentes necessidades e avaliar os resultados e obstáculos que possam vir a surgir nesses outros contextos. Busca-se ainda investigar a relação entre essas tecnologias e os ganhos de aprendizagem que elas podem proporcionar aos alunos.
%


\section{Trabalhos Futuros}

Acredita-se que a melhor forma de continuar esse trabalho é utilizando o sistema
desenvolvido em um turma por um longo período para assim descobrir e desenvolver outros requistios que não
foram considerados nesse trabalho. Além disso, pode-se fazer um estudo de caso para verificar os benefícios
do sistema na sala de aula, ou seja, aqueles que foram descritos no \autoref{chap:revision}, \autoref{sec:beneficios},
como aumento na frequência escolar, melhora na atenção, engajamento da turma, etc.
