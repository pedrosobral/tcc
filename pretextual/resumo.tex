% resumo em português
\setlength{\absparsep}{18pt} % ajusta o espaçamento dos parágrafos do resumo
\begin{resumo}
  Este trabalho apresenta um breve estudo sobre o aprendizado ativo

  Nesse trabalho de conclusão de curso será desenvolvido um sistema de resposta em sala de aula,
  que possibilite aos professores e aos estudantes uma ferramenta de software livre,
  que permita usar {\textit{smartphones}} como {\clickers} para que o mesmo possa
  ser usado principalmente com práticas pedagógicas de aprendizado ativo como o \textit{Peer Instruction}.
 % O resumo deve ressaltar o
 % objetivo, o método, os resultados e as conclusões do documento. A ordem e a extensão
 % destes itens dependem do tipo de resumo (informativo ou indicativo) e do
 % tratamento que cada item recebe no documento original. O resumo deve ser
 % precedido da referência do documento, com exceção do resumo inserido no
 % próprio documento. (\ldots) As palavras-chave devem figurar logo abaixo do
 % resumo, antecedidas da expressão Palavras-chave:, separadas entre si por
 % ponto e finalizadas também por ponto.

 \textbf{Palavras-chave}:  Aprendizado Ativo. Instrução pelos Colegas. Sistemas de Resposta em Sala de Aula. Aplicativos Híbridos para Celular.
\end{resumo}

% resumo em inglês
\begin{resumo}[Abstract]
 \begin{otherlanguage*}{english}


   \vspace{\onelineskip}

   \noindent
   \textbf{Keywords}: Active Learning. Peer Instruction. Classroom Response Systems. Hybrid Mobile Applications.
 \end{otherlanguage*}
\end{resumo}
