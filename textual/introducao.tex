\chapter{Introdução}

Assim como o artista propõe a sua obra ao público, assim deve ser o professor que
propõe conhecimento aos seus estudantes. No entanto, hoje parece que ainda prevalece
o modelo em que o professor é o transmissor de informações em aulas puramente expositivas,
em que prevalece a baixa participação dos alunos \cite[p. 8]{Silva2001}.

Por que ainda se justifica o modelo de ensino em que se baseia
na disseminação de informações, que nunca foi tão fácil achar (internet,
livros, etc)? O ensino de hoje deveria estar focado para uma nova
visão em que o papel do professor seja de intermediar o ensino \cite[p. 19]{Araujo2013}.

Sistemas de respostas são sistemas que possibilitam que todos os alunos
respondam a questões apresentadas no projetor. Geralmente um gráfico de barras
é apresentado logo em seguida que os estudantes submetem as suas soluções
utilizando algum dispositivo remoto. As respostas são anônimas para os seus colegas,
no entanto o professor pode identificar cada estudante individualmente pela
identificação única do dispositivo, permitindo assim uma análise individual \cite[p. 1]{Kay2009}.


Esse resultado imediato pode ser usado juntamente
com metodos que promovem a interação social voltada para a aprendizagem como a
Instrução pelos Colegas (IpC) que tem alcançado sucesso internacionalmente \cite[p. 3]{Araujo2013}.

\section{Objetivo Geral}
Desenvolver um sistema de resposta em sala de aula,
que possibilite aos professores e aos estudantes uma ferramenta de software livre,
que permita usar {\textit{smartphones}} como {\clickers} para que o mesmo possa
ser usado principalmente com práticas pedagógicas de aprendizado ativo como o IpC.

\section{Objetivos Específicos}

\begin{itemize}
    \item Realizar levantamento de requisitos sobre os sistemas de resposta em sala de aula;
    \item Especificar e implementar uma aplicação para dispositivos móveis, que será utilizado como {\clickers};
    \item Especificar e implementar uma aplicação web para o professor administrar as questões e gerar relatórios;
    \item Especificar e implementar um sistema servidor, para receber e
    enviar dados para os os clientes: dispositivos móveis dos alunos e navegador
    web do professor.
\end{itemize}

\section{Organização do texto}
Além desta introdução, este trabalho está dividido em mais dois capítulos.

\begin{description}
  \item[Revisão de literatura:] Esse capítulo discute um contexto pedagógico
  para o uso de sistemas de resposta em sala de aula. Inicialmente, é
  apresentado o conceito de aprendizado ativo e um método de implementação: Instrução pelos Colegas.
  Em seguida, são apontados os sistemas de resposta em sala de aula como uma
  ferramenta para ajudar o professor a mediar um aprendizado significativo  em
  sala de aula. O capítulo é encerrado apresentando os benefícios e desafios de uso dessa tecnologia.

  \item[Material e Métodos:] Nesse capítulo serão descritas as fases de especificação e projeto do sistema.
  Ainda nesse capítulo são apresentadas as ferramentas e a arquitetura
  do sistema, dando uma visão geral do que foi desenvolvido.

\end{description}
