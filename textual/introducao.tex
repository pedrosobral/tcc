\chapter{Introdução}

Por que ainda se justifica o modelo de ensino em que se baseia
na disseminação de informações, que nunca foi tão fácil achar (internet,
livros, etc)? O ensino de hoje deveria estar focado para uma nova
visão em que o papel do professor seja de intermediar a aprendizagem \cite{Araujo2013}.

Sistemas de respostas para uso em sala de aula são sistemas que possibilitam que todos os alunos
respondam a questões apresentadas no projetor. Geralmente um gráfico de barras
é apresentado logo em seguida que os estudantes submetem as suas soluções
utilizando algum dispositivo remoto também conhecido como \textit{clickers}. As respostas são anônimas para os seus colegas,
no entanto o professor pode identificar cada estudante individualmente pela
identificação única do dispositivo, permitindo assim uma análise individual \cite{Kay2009}.

O resultado imediato disponibilizado por tais sistemas pode ser usado juntamente
com metodos que promovem a interação social voltada para a aprendizagem como a
Instrução pelos Colegas (IpC), que tem alcançado sucesso internacionalmente \cite{Araujo2013}.

Não existe na literatura um consenso sobre a nomenclatura para referenciar sistemas de
resposta para uso em sala de aula. Pode-se encontrar termos como
{\textit{``student response system''}} (sistema de resposta do estudante),
{\textit{``audience response system''}} (sistema de resposta a audiência),
{\textit{``personal response systems''})} (sistema de resposta pessoal),
{\textit{``classroom response systems''} (CRS)} (sistemas de resposta em sala de aula),
ou apenas como {``\clickers''} \cite{Hunsu2016}.
Nesse trabalho, para simplificação foi utilizado apenas a sigla {\clickers} {\textit{``Classroom Response System''}
para referenciar esse tipo de sistema.

\section{Objetivo Geral}
Desenvolver um sistema de resposta para uso sala de aula,
que possibilite aos professores e aos estudantes uma ferramenta de software livre,
que permita usar {\textit{smartphones}} como \textit{clickers} para que o mesmo possa
ser usado principalmente com práticas pedagógicas de aprendizado ativo como o IpC.

\section{Objetivos Específicos}

\begin{itemize}
    \item Realizar levantamento de requisitos sobre os sistemas de resposta em sala de aula;
    \item Especificar e implementar uma aplicação para dispositivos móveis, que será utilizado como \textit{clickers};
    \item Especificar e implementar uma aplicação web para o professor administrar as questões e gerar relatórios;
    \item Especificar e implementar um sistema servidor, para receber e
    enviar dados para os os clientes: dispositivos móveis dos alunos e navegador
    web do professor.
\end{itemize}

\section{Organização do texto}
Além desta introdução, este trabalho está dividido em mais quatro capítulos.

\begin{description}
  \item[Revisão de literatura:] Esse capítulo discute um contexto pedagógico
  para o uso de sistemas de resposta em sala de aula. Inicialmente, é
  apresentado o conceito de aprendizado ativo e um método de implementação: Instrução pelos Colegas.
  Em seguida, são apontados os sistemas de resposta em sala de aula como uma
  ferramenta para ajudar o professor a mediar um aprendizado significativo  em
  sala de aula. O capítulo é encerrado apresentando os benefícios e desafios de uso dessa tecnologia.

  \item[Engenharia de Software:] Nesse capítulo são descritas as fases de especificação, projeto, implementação e testes do sistema.
  Os procedimentos, material e métodos, resultados e discussões e cada etapa também é apresentado nesse capítulo.

  \item[Resposta em Sala de Aula:] Nesse capítulo é apresentado o software desenvolvido,
  definições, as principais telas e característica do sistema desenvolvido.

  \item[Considerações finais e trabalhos futuros:] São realizadas as considerações finais do
  trabalho e apontadas diretrizes para trabalhos futuros.

\end{description}
